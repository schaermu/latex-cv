%% Copyright 2006-2015 Xavier Danaux (xdanaux@gmail.com).
%
% This work may be distributed and/or modified under the
% conditions of the LaTeX Project Public License version 1.3c,
% available at http://www.latex-project.org/lppl/.

\documentclass[11pt,a4paper,sans]{moderncv}
\moderncvstyle{fancy}
\moderncvcolor{blue}
\nopagenumbers{}
\usepackage[utf8]{inputenc}
\usepackage[scale=0.75]{geometry}

% personal data
\name{Stefan}{Schärmeli}
\title{Lebenslauf}
\address{Winkelriedstrasse 46}{3014 Bern}{Schweiz}
\phone[mobile]{+41763686616}
\email{info@schaermu.ch}
\homepage{www.schaermu.ch}
\social[linkedin]{stefan.schaermeli}
\social[github]{schaermu}
\photo[64pt][0pt]{picture}

\makeatletter\renewcommand*{\bibliographyitemlabel}{\@biblabel{\arabic{enumiv}}}\makeatother

\begin{document}

\makecvtitle

\section{Beruflicher Werdegang}
\cventry{11/2016--heute}{Senior Web Developer}{smartfactory GmbH}{Biel}{}{Bei der Firma smartfactory werden Mobile-Apps, Websites und AR/VR-Applikationen entwickelt. Ich zeichnete für diverse Themen verantwortlich, bewegte mich jedoch vor allem im Bereich App-CMS Entwicklung.\newline{}%
Abgesehen von Web-Entwicklung kümmerte ich mich u.A. um folgende Themen:%
\begin{itemize}%
\item Continuous Integration/Deployment
\item Monitoring/Betrieb der Webserver (Ubuntu)
\item Standardisierung/Einführung von firmeninternen Buildpipelines
\end{itemize}}

\cventry{07/2012--07/2016}{Senior Web Developer}{Maxomedia AG}{Bern}{}{Implementierung, Planung und Wartung von CMS-Websites, Plattformen und App-Backends in ASP.NET MVC. Zusätzlich Betrieb und Konfiguration von Microsoft-Webservern und SQL-Servern sowie die fachliche Führung des Teams Digital.\newline{}Ab 2016 Team-Leiter Team Digital, personelle Führung.}

\cventry{08/2011--07/2012}{Application Manager}{Interactive Data Managed Solutions}{Zürich}{}{Überwachung und Planung von Software-Deployments, Koordination und Enforcement des ITIL-Prozesses und Ticket-Management.}

\cventry{12/2007--06/2011}{Web Developer}{neue chemigraphie AG}{Urdorf}{}{Implementierung und Wartung von Druckvorstufe-Automatisierungen und Web-Applikationen zur Verwaltung von Mediendatenbanken. Unterhalt und Wartungsaufgaben auf Linux-Infrastruktur.}

\cventry{12/2007--06/2011}{selbstständig/Freelancer}{}{}{}{Implementierung von Custom- sowie Typo3-PHP Websites, später PHP Auftragsarbeiten sowie Server-Betreuung im Linux-Umfeld.}

\clearpage

\section{Ausbildung}
\cventry{08/1999--07/2003}{Informatiker EFZ}{GIBB Bern / SBB}{Bern}{}{Abschluss als Informatiker EFZ}

\section{Sprachen}
\cvitem{Deutsch}{Muttersprache}
\cvitem{Englisch}{mündlich und schriftlich sehr gut}
\cvitem{Französisch}{Schulkenntnisse}

\section{Skillset}
\cvitemwithcomment{Javascript}{ausgezeichnet}{ES5/ES6, node.js, AngularJS, Vue.JS}
\cvitemwithcomment{C\#/ASP.NET}{sehr gut}{ASP.NET MVC/WebAPI 2, EntityFramework}
\cvitemwithcomment{Python}{sehr gut}{django, rest-framework}
\cvitemwithcomment{Buildpipelines}{sehr gut}{Webpack, Gitlab CI, Fastlane, Gradle}
\cvitemwithcomment{DevOps}{(sehr) gut}{GitLab CI, Docker, Chef/Puppet}
\cvitemwithcomment{HTML5/CSS3}{(sehr) gut}{SASS, Stylus}
\cvitemwithcomment{PHP}{gut}{Laravel/Wordpress}

\section{Freizeit}
\cvitem{Joggen / Schwimmen}{...oder auch "den hohen Steuersatz in Bern mit einem Aareschwumm erträglicher machen".}
\cvitem{Filme / Bücher}{Über gute (und auch schlechte) Filme oder Bücher könnte ich mich stundenlang unterhalten.}
\cvitem{Open-Source}{Sofern die Zeit es zulässt, involviere ich mich gerne in spannenden Open-Source Projekten.}

\section{Referenzen}
\begin{cvcolumns}
  \cvcolumn{smartfactory Gmbh}{\begin{itemize}\item Roger Lüchinger, Geschäftsführer{\newline}+41 32 365 30 00\end{itemize}}
  \cvcolumn{Weitere}{\begin{itemize}\item Auf Anfrage\end{itemize}}
\end{cvcolumns}
\clearpage

%-----       Motivationsschreiben
\recipient{Schweizerische Post AG}{Toni Stucki\\Teamleiter Innovation \& Tools}
\date{25. Mai 2018}
\opening{Sehr geehrte Damen und Herren,}
\closing{Mit freundlichen Grüssen,}
\enclosure[Beilage]{Lebenslauf{}}
\makelettertitle

Ich bin über Stackoverflow auf ihr spannendes Stellen-Inserat "Full Stack Developer" gestossen und fühlte mich sofort angesprochen, dies aus 2 Hauptgründen.

Der erste Aspekt ist die Technologie: seit 15 Jahren arbeite ich nun im Bereich Web-Entwicklung, einem Bereich welcher sich sehr schnell fortbewegt und eine hohe Innovationsgeschwindigkeit hat. Ich konnte mich über diese Zeit vor allem dank meines Wissenshungers behaupten, neue Technologien schüchtern mich nicht ein, sie treiben mich an.

Der zweite, für mich persönlich wichtigere Aspekt ist das in der Ausschreibung beschriebene Wirkungsfeld. Während meiner Zeit in einer Werbeagentur arbeitete ich ebenfalls mit Textern, UX-Spezialisten und Grafikern zusammen. Dieser Aspekt des Jobs hat sehr stark meinem Wesen entsprochen und fehlt mir heute. In einem Innovations-Umfeld mit Ideengebern und anderen Spezialisten in kleinen, agilen Teams und zeitlich begrenzten Formaten Prototypen erstellen und validieren? Das hört sich für mich wie ein Traumjob an.

Der letzte Grund (last but not least) für meine Bewerbung sind Sie, die schweizerische Post. Die Möglichkeiten und Perspektiven, die Diversität der Geschäftsfelder und die attraktiven Arbeitsbedingungen sprechen ihre eigene Sprache.
{\newline}{\newline}{\newline}
Ich würde mich sehr über ein beldiges Gespräch mit Ihnen freuen.
{\newline}{\newline}{\newline}
\makeletterclosing

\end{document}
